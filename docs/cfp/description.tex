\documentclass[12pt,oneside]{article}
% \usepackage[T2A]{fontenc} 
\usepackage{cmap}
\usepackage[utf8]{inputenc}
\usepackage[russian]{babel}

\usepackage[a4paper, left=30mm, top=20mm, right=20mm, bottom=20mm, nohead]{geometry} % Поля

\usepackage{url}

\title{Единый каталог научных мероприятий}
\author{}

\begin{document}
\maketitle

\section{Цель работы}

\paragraph{Описание проблемы.} Ежегодно в мире проводятся десятки тысяч научных мероприятий разного масштаба и научного уровня (от однодневных семинаров по узкой области до всемирных конгрессов по направлению науки). Активно работающие ученые обычно имеют представление о наиболее важных мероприятиях в своей области, но часто возникает необходимость найти подходящее мероприятие в смежной дисциплине. При этом не всегда очевидно, какое из предстоящих мероприятий будет полезно с научной точки зрения. Существуют тематические каталоги предстоящих мероприятий, но они не дают возможность оценить качество мероприятий, не полны и не позволяют вести поиск по тематики или другим параметрам.

\paragraph{Цель.} Создание единой базы данных анонсов научных мероприятий с возможностью поиска по тематике и предполагаемому научному уровню.

\paragraph{Аудитория.} Научные сотрудники, студенты и аспиранты, организаторы, наукометрические приложения.

\paragraph{Конкуренты.} COMS\footnote{\url{https://conference-service.com/}}.

\section{Задачи}

\subsection{Сбор первичных данных и извлечение информации}
\paragraph{Поиск каталогов.} Каталог~--- это страница, содержащая ссылки на страницы мероприятий. Каталоги могут поддерживаться отдельными учеными, научными лабораториями или институтами\footnote{\url{https://faculty.math.illinois.edu/~west/meetlist.html}}. Количество каталогов в таком понимании может быть велико, а количество ссылок в каждом из них~--- нет. Метод поиска новых каталогов может быть основан на анализе обратных ссылок. Страница каталога должна быть в выдаче поисковой системы по запросу <<кто ссылается на страницу данной конференции>>.
\marginpar{Ю.Г.}

\paragraph{Извлечение минимального набора метаданных из записи каталога.}
Запись о конференции должна содержать название или аббревиатуру конференции, даты и место проведения, ссылку на домашнюю страницу. Необходимо реализовать систему извлечения указанных данных, удовлетворяющую следующим требованиям: автоматическое распознавание структуры HTML документа, автоматический контроль качества извлечения.
\marginpar{Ю.Г.}

\paragraph{Классификация гипертекстовых ссылок <<конференция-нет>>.} Для заданного адреса URL требуется определеить, является ли данная страница или ресурс описанием научной конференции.

\paragraph{Извлечение полного описания со страницы конференции.}
Анонс предстоящей конференции\footnote{\url{https://listserv.acm.org/SCRIPTS/WA-ACMLPX.CGI?A2=MOD-DBWORLD;9c3a8ca5.2110}} может содержать следующую информацию. Название, аббревиатура, тап мероприятия, место, страна и сроки проведения, перечень тематик конференции, сроки подачи тезисов и текстов докладов, срок извещения о результатах рассмотрения, список членов программного комитета с указанием места работы, список принятых докладов. Требуется реализовать систему извлечения этой информации из текста анонса или со страницы сайта мероприятия.

\subsection{Тематический анализ}
Источник данных: научные направления из анонса, научные направления членов программного комитета, совместное цитирование.
\paragraph{Определение тематики мероприятия.}
\paragraph{Выделение близких по тематике мероприятий.}
\paragraph{Определение тематик набора публикаций.}
\paragraph{Излечение ключевых слов из аннотаций.}
Разработать нейронную четь, которая извлекает из текста аннотации научной статьи словосочетания, обладающие признаками ключевыз слов.
\marginpar{АЧ?}
\paragraph{Построение онтологии научных исследований.}
Для заданной метаонтологии, включающей такие понятия как \emph{задача} и \emph{метод решения}, разработать метод классификации ключевых слов по этой системе понятий.
\marginpar{АЧ?}

\subsection{Вычисление количественных характеристик для оценки научного уровня мероприятия}
\paragraph{Выделение серий конференций.}
\paragraph{Нахождение соответствий между названиями докладов и публикациями в журналах.}
Источник публикаций~--- CrossRef.
\paragraph{Нахождение соответствий между конференциями и специальными выпусками журналов.}

\paragraph{Построение профиля ученого.} На основании данных внешней библиографической базы данных (Scopus, Crossref, и т.п.) требуется построить профиль указанного автора. В профиль входят публикации, области научных интересов (рубрики классификатора), сведения о месте работы.

\paragraph{Оценка научного уровня членов программного комитета.}
На основании данных профиля и с использованием внешней системы цитирования (Scopus, Web of Science).  Определение области научных интересов автора, опеределение степени соответствия его интересов и тематики мероприятия, оценка формальных показателей автора, оценка аналогичных показателей других авторов в этой области. 

\paragraph{Определение <<однородности>> тематики мероприятия.}

\paragraph{Оценка охвата научных коллективов участниками мероприятия.} Существуют конференции, на которые регулярно приезжают одни и теже участники. Результаты цитируются только внутри этой группы.


\subsection{Поиск}
\paragraph{Поиск мероприятий по набору ключевых слов.}
\paragraph{Уточнение поиска, интерактивный поиск.}
\paragraph{Поиск по параметрам.} Наличие лекций для студентов, продолжительность доклада, объем тезисов и т.п.


\subsection{Алгоритмы анализа данных}
\paragraph{Ведущие ученые и группы.}\marginpar{ДР}
\paragraph{Выделение новых направлений или новых междисциплинарных исследований.}
\paragraph{Построение персональных рекомендаций участникам.}


\subsection{Реализация}
\paragraph{Web-приложение.}
\paragraph{Web-приложение организатора.}
Планирование программы (OptaPlanner).
\paragraph{Мобильное приложение участника.} Просмотр программы, аннотаций докладов, составление собственного расписания, уведомления организаторов.




\end{document}
